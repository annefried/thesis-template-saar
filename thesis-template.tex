% % % % % % % % % % % % % % % % % % % % %
% Saarland Ph.D. thesis template		%
% created by Annemarie Friedrich		%
% % % % % % % % % % % % % % % % % % % % %

% % % % % % % % % % % % % % % % % % %
% Layout and fonts					%
% % % % % % % % % % % % % % % % % % %

\usepackage{fancyhdr}
\usepackage{graphicx}

% Nice sans serif font
\renewcommand{\encodingdefault}{T1}
\usepackage[scaled=0.9]{helvet}


% Paragraph and indent (as required by Prof. Dr. Pinkal)
\setlength{\parindent}{0pt}
% spacing between lines
\linespread{1.1}
\setlength{\parskip}{0.4ex plus 0.3ex minus 0.1ex}
% have nice hanging captions for figures etc.
\usepackage[labelfont=bf,format=hang]{caption}
% this needs to be before fancy header
\usepackage[includeheadfoot,a4paper,total={6in, 8.5in}]{geometry}

%%%%%%%%%%%%%%%%%%%%%%%%%%%%%%
%% fancy header			    %%
%%%%%%%%%%%%%%%%%%%%%%%%%%%%%%

\pagestyle{fancyplain}
\renewcommand{\chaptermark}[1]{%
	\markboth{\thechapter.\ #1}{}}
\lhead[\fancyplain{}{\thepage}]%
{\fancyplain{}{\bfseries \sffamily \nouppercase \leftmark}}
\rhead[\fancyplain{}{\bfseries \sffamily \nouppercase \leftmark}]%
{\fancyplain{}{\thepage}}
\cfoot{}

% TODO: header for appendix?


%%%%%%%%%%%%%%%%%%
%%  title page  %%
%%%%%%%%%%%%%%%%%%

% Adapted from a template by Robert Dahlke and Sigmund Stintzing (LMI München 2002)

\usepackage{german}

\newcommand{\ThesisTitle}[9]{
  \thispagestyle{empty}
  \vspace*{\stretch{1}}
  {\parindent0cm
  \rule{\linewidth}{.7ex}}
  \begin{flushright}
    \vspace*{\stretch{1}}
    \sffamily\bfseries\Huge
    #1\\
    \vspace*{\stretch{2}}
    \sffamily\bfseries\large
    #2
    \vspace*{\stretch{1}}
  \end{flushright}
  \rule{\linewidth}{.7ex}
  
  \vspace*{1cm}
  
  \begin{center}
	    \includegraphics[width=1.5in]{owl}
  \end{center}

  \vspace*{\stretch{1}}
  \begin{center}
    \Large Dissertation\\
    \Large zur Erlangung des akademischen Grades\\
    \Large eines Doktors der Philosophie\\
    \Large an den Philosophischen Fakult\"aten\\
    \Large der Universit\"at des Saarlandes\\
     \vspace*{0.5cm}
    \large vorgelegt von\\
    \large #2\\
    \large aus #3\\
    \vspace*{0.5cm}
    \large  \sffamily Saarbr\"ucken, den #4
  \end{center}

  \newpage
  \thispagestyle{empty}

  \vspace*{\stretch{1}}

  \begin{flushleft}
  	\large Dekan der Philosophischen Fakult\"at II: #5 \\[1mm]
    \large Erstgutachter:  #6 \\[1mm]
    \large Zweitgutachter: #7 \\[1mm]
    \large Drittgutachter: #8 \\[1mm]
    \large Tag der m\"undlichen Pr\"ufung: #9\\
  \end{flushleft}

  \cleardoublepage
}


% % % % Chapter headings % % % % %
\usepackage{titlesec}
\titleformat{\chapter}[display]
{\normalfont\sffamily\LARGE\bfseries}
{\chaptertitlename\ \thechapter}{0pt}{\rmfamily\Huge\rule{\linewidth}{.1ex}\\}[\vspace*{-0.6cm}\rule{\linewidth}{.1ex}]

\titleformat{\part}[display]
{\normalfont\sffamily\Huge\bfseries\centering}
{\partname\ \thepart\\ \rule{\linewidth}{.1ex}}{10pt}{}


% % % % % % % % % % % % % % % % % % %
% Language stuff					%
% % % % % % % % % % % % % % % % % % %
\usepackage[utf8]{inputenc}
%\usepackage[T1]{fontenc} % adding this makes sf font look weird??
\usepackage[english]{babel}



% % % % % % % % % % % % % % % % % % %
% citation							%
% % % % % % % % % % % % % % % % % % %
\usepackage[round]{natbib}
\usepackage[toc,page]{appendix}

% % % % % % % % % % % % % % % % % % %
% graphics etc						%
% % % % % % % % % % % % % % % % % % %
\usepackage{graphicx}
\usepackage[usenames,dvipsnames]{xcolor}
% todonotes: useful while writing!
\usepackage[backgroundcolor=yellow!30]{todonotes}
% for showing todonotes properly
\setlength{\marginparwidth}{2.2cm}
%\reversemarginpar

% % % % % % % % % % % % % % % % % % %
% useful custom commands			%
% % % % % % % % % % % % % % % % % % %

% dense underline
\newcommand{\dul}[1]{\underline{\smash{#1}}}


% % % % % % % % % % % % % % % % % % %
% stuff to create nice tables		%
% % % % % % % % % % % % % % % % % % %
\usepackage{booktabs} % For \toprule, \midrule and \bottomrule
\usepackage{siunitx} % Formats the units and values
\usepackage{pgfplotstable} % Generates table from .csv
\usepackage{multirow}
\usepackage{longtable} % for tables breaking pages
\usepackage{array}
\usepackage{ragged2e}
\newcolumntype{P}[1]{>{\RaggedRight\hspace{0pt}}p{#1}}
\usepackage{adjustbox}
\usepackage{array}
\newcolumntype{R}[2]{%
	>{\adjustbox{angle=#1,lap=\width-(#2)}\bgroup}%
	l%
	<{\egroup}%
}
\newcommand*\rot{\multicolumn{1}{R{45}{1em}}}% no optional argument here, please!


% % % % % % % % % % % % % % % % % % %
% environments						%
% % % % % % % % % % % % % % % % % % %
\usepackage{url}
\usepackage{slashbox}
\usepackage{indent}
\usepackage{textcomp}
\usepackage{enumerate}
\usepackage{verbatim} % for comments
\usepackage{fancyvrb}
\usepackage{listings} % for comments
\usepackage{qtree} % for syntactic trees (images)


\newenvironment{packed_enum}{
	\begin{enumerate}
		\setlength{\itemsep}{1pt}
		\setlength{\parskip}{0pt}
		\setlength{\parsep}{0pt}
		\setlength{\leftmargin}{0pt}
	}{\end{enumerate}}

\newenvironment{packed_enum_more_indent}{
	\begin{enumerate}
		\setlength{\itemsep}{1pt}
		\setlength{\parskip}{0pt}
		\setlength{\parsep}{0pt}
		\setlength{\leftmargin}{4pt}
	}{\end{enumerate}}

\newenvironment{packed_item}{
	\begin{itemize}
		\setlength{\itemsep}{1pt}
		\setlength{\parskip}{0pt}
		\setlength{\parsep}{0pt}
		\setlength{\leftmargin}{0pt}
	}{\end{itemize}}

\newenvironment{packed_list}{
	\begin{list}
		\setlength{\itemsep}{1pt}
		\setlength{\parskip}{0pt}
		\setlength{\parsep}{0pt}
		\setlength{\leftmargin}{0pt}
	}{\end{list}}

\setlength{\fboxsep}{0.3cm}
\setlength{\fboxrule}{1pt}

% Example environment (for language examples)
% use with \begin{example}
%... 
%\end{example}

\newcounter{examplecounter}
\newlength\myLeftmargin
\newenvironment{example}
{\refstepcounter{examplecounter}
\begin{indentation}{2.5em}{2em}
	\setlength{\parsep}{0pt}
	\setlength{\parskip}{0pt}
	\setlength{\itemsep}{0pt}
	\begin{list}{\textbf{(\arabic{examplecounter})}}%
		{\global\addtolength\myLeftmargin{\parindent}%
			\global\addtolength\myLeftmargin{\parindent}
			\setlength\leftmargin{\myLeftmargin}%
		}
		\item\relax}
	{\end{list}
\end{indentation}}
	
	
\newcommand{\labex}[1]{\label{ex:#1}}
	
% modular references
\newcommand{\eref}[2][]{(\ref{ex:#2}#1)} %examples
\newcommand{\aref}[3][]{(\ref{ex:#2}#1#3)} %examples with (a) / (b)
\newcommand{\cref}[1]{Chapter~\ref{chap:#1}} % chapters
\newcommand{\dref}[1]{Definition~\ref{def:#1}} % definitions
\newcommand{\tref}[1]{Table~\ref{tab:#1}} % tables
\newcommand{\fref}[1]{Figure~\ref{fig:#1}} % figures
\newcommand{\sref}[1]{Section~\ref{sec:#1}} %sections
